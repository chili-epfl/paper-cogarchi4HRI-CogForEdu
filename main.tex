
\documentclass[conference]{IEEEtran}
% *** CITATION PACKAGES ***
%
\usepackage{cite}

% *** GRAPHICS RELATED PACKAGES ***
%
\usepackage[pdftex]{graphicx}
\graphicspath{./figures/}

% *** SUBFIGURE PACKAGES ***
%\ifCLASSOPTIONcompsoc
%  \usepackage[caption=false,font=normalsize,labelfont=sf,textfont=sf]{subfig}
%\else
%  \usepackage[caption=false,font=footnotesize]{subfig}
%\fi

\usepackage{url}


\begin{document}

\title{Cognitive Architecture For Learning with Robots}


% author names and affiliations
% use a multiple column layout for up to three different
% affiliations
\author{\IEEEauthorblockN{A}
\IEEEauthorblockA{address A\\
address A\\
Email: todo}
\and
\IEEEauthorblockN{Homer Simpson}
\IEEEauthorblockA{Twentieth Century Fox\\
Springfield, USA\\
Email: homer@thesimpsons.com}
\and
\IEEEauthorblockN{James Kirk\\ and Montgomery Scott}
\IEEEauthorblockA{Starfleet Academy\\
San Francisco, California 96678--2391\\
Telephone: (800) 555--1212\\
Fax: (888) 555--1212}}



\maketitle

% As a general rule, do not put math, special symbols or citations
% in the abstract
\begin{abstract}
Robots new application

One of this new application, co-learner


In the context of learning with a robot, we discuss in this paper  the necessity for robots to be able of cognitive reasoning. 
We present the context of learning assisted by robot. 
We then argue for cognitive architectures able to assure social interaction in this type of context. 

Current cog architecture not tailored for social HRI and to model mental states from social behaviour w humanoid robot

Few applied in this context

We introduce some requirements for such architecture. 
We finally present a proposal of architecture based on mutual modelling. 

Our architecture aims to contribute to the SOA by \dots

\end{abstract}

% no keywords




% For peer review papers, you can put extra information on the cover
% page as needed:
% \ifCLASSOPTIONpeerreview
% \begin{center} \bfseries EDICS Category: 3-BBND \end{center}
% \fi
%
% For peerreview papers, this IEEEtran command inserts a page break and
% creates the second title. It will be ignored for other modes.
\IEEEpeerreviewmaketitle



\section{Introduction}
Why should you use cognitive architectures - how would they benefit your research as a theoretical framework, a tool and/or a methodology?

Should cognitive architectures for social interaction be inspired and/or limited by models of human cognition?

What are the functional requirements for a cognitive architecture to support social interaction?

How the requirements for social interaction would inform your choice of the fundamental computational structures of the architecture (e.g. symbolic, sub-symbolic, hybrid, ...)?


Can you devise a social interaction scenario that current cognitive architectures would likely fail, and why?


Robots present as assistant in various domains health, rescue, education

Robots are brought o be in social interaction with humans.
Necessity to behave in an acceptable and understandable manner by the user.

describe research context, 


robots for education

learning by teaching

co-writer

\section{The need for cog archi in educational robotics}
way to represent the learners knowledge

essential for logical reasoning, 
explanation of concepts



\section{Related Works}
According to \cite{Vernon2011}, the term ``cognitive architecture'' aimed to unify theories of cognition on various issues such as attention, memory, problem solving, decision making, learning, across different disciplines including psychology, neuroscience and computer-sciences.
Some architectures are based on philosophical theories, some on biological theories and others on psychological theories~\cite{Thorisson2012,Chong2009}.
Often; they do not aim to tackle the same research problem. 
For instance, often, biologically based cognitive architectures aim to mimic the human brain while psychological ones focus on cognitive processes.

There exist two main families of cognitive architectures: the \textit{cognitivist} and the \textit{emegergentist} perspectives.

\paragraph{The Emergent Perspective} covers developmental cognitive architectures.
In these types of architectures, the model and the processes are learned from experience. 
Thus, the knowledge is automatically acquired. 
However, these architectures also assume a part of innate knowledge at first.
These systems are often platform-sensitive but research here usually focuses on general frameworks of model acquisition in order to be reusable on other platforms. 
The model and the process are often task and domain dependent and are linked to sensorimotor loops.
The emergent cognitive architectures reflect in some way the morphology of the system. 
In this category, we find for example the \textit{HAMMER} \cite{Demiris2005} or the \textit{ICub Cognitive architecture} \cite{Vernon2007}.


\paragraph{The Cognitivist Perspective} considers the cognitive architectures as generic computational models neither domain, nor task-specific. 
A human programmer or machine learning feed the system with knowledge making it more specific to a task or a domain.
The system is composed of computational models of cognition that are  taken from various sources (i.e biology, psychology, philosophy).
For the cognitivists, the cognitive processes are independent from the physical platform (increasing the generic aspect of the computational models).

\textit{ICARUS}, belonging to this family of cognitive architectures \cite{LaChRo2005,Langley2006} is also grounded in cognitive psychology, and \textit{AI} (like \textit{BDI} \footnote{BDI : Belief, Desire, Intention logic, developed by \cite{bratman1987}}) aims at unifying reactive and deliberative (problem-solving) execution, as well as symbolic and numeric (utilities) reasoning. 
Memory is organised into short vs long-term, and conceptual vs skill memory.
ICARUS has several goals but focuses only on the one with highest unsatisfied priority.
The skills that allow to achieve it are brought from long-term to short-term memory.
If no skill is available, the system uses means-end analysis to decompose into sub-goals, and learns from this impasse.

For instance, some architectures are based on a set of generic and symbolic rules, such as \textit{Soar} \cite{Laird2012}, based on the unified theory of cognition, or \textit{ACT-R} \cite{Anderson2005}. 
Many of these architectures are based on the ``mind-is-like-a-computer'' analogy.

\begin{table}
	\begin{tabular}{p{0.5\linewidth} p{0.5\linewidth}} \hline
		System 1 (old mind) (intuitive) & System 2 (new mind) (reflective) \\ \hline
		Does not require working memory Autonomous &
		Requires working memory Cognitive decoupling; mental simulation \\
		Fast & Slow \\
		High capacity & Capacity limited \\
		Parallel & Serial \\
		Non-conscious &	Conscious \\
		Biased responses & Normative responses \\
		Contextualized & Abstract \\
		Automatic & Controlled \\
		Associative & Rule-based \\
		Experience-based decision making & Consequential decision making \\
		Independent of cognitive ability & Correlated with cognitive ability \\ \hline
		Evolved early  & Evolved late \\
		Similar to animal cognition &	Distinctively human \\
		Implicit knowledge  & Explicit knowledge \\
		Basic emotions & Complex emotions \\ \hline
	\end{tabular}
	\caption{Properties of System 1 and System 2 from the Dual-Process Theories of cognition from \cite{Evans2013}.}
	\label{fig:dual_system}
\end{table}
%\vspace{1cm}

The Dual-process theory marries these two families of cognitive architectures \cite{Evans2013}: the cognitivist and the emergentist.
This theory states that there exist two cognitive systems involved in cognition, one fast and intuitive (S1) and the other, slow and deliberative (S2).
The table~\ref{fig:dual_system} summarizes different characteristics of these two subsystems.
Underlying these two systems (S1 and S2), we see the two families of cognitive architectures (Emergent and Cognitivist).

\textbf{Hybrid approaches} of cognitive architectures constitute the last family combining Emergent and Cognitivist approaches.
Hybrid architectures combine both types of processing. \textit{CLARION}  \cite{Sun2003} is an example of such an architecture.
Recently, other architectures have been proposed, focusing on memory \cite{baxter2013} or attentional processes and a sensori-motor approach \cite{demiris2006}.


\section{Mutual Modelling towards a model for TOM}
Subsubsection text here.



%What is the primary outstanding challenge in developing and/or applying cognitive architectures to social HRI systems?
\section{Challenges for Cognitive Architecture for Educational HRI}
Mainly method of evaluation to prove the following concepts
\begin{enumerate}[noitemsep]
		\item Categorise and Understand: enable to build ontologies and to acquire semantic knowledge.
		\item Have an episodic memory and reflective processes.
		\item Use ontology in order to encode the knowledge for more flexibility and re-usability.
		\item Enable to communicate decision and plans.
		\item Integrate new sensors for physical perception of the world.
		\item Enable embodied applications and expressions.
		\item Integrate emotions in the cognitive process.
		\item Enable generalisation and modularity.
\end{enumerate}


\section{Conclusion}
The conclusion goes here.


\section*{Acknowledgment}


The authors would like to thank...






\bibliographystyle{IEEEtran}

\bibliography{biblio}

\end{document}


